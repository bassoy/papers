% 2D-Plots mit Durchschnittswerten gemittelt über die Tensorgröße und Kontraktionsmodus
\begin{tikzpicture}
\begin{axis}[height=0.25\textheight,width=0.5\textwidth, style={font=\footnotesize}, grid=major, grid style={dotted}, align=center, xlabel={Order}, ylabel={Gflops/s}, xlabel near ticks, ylabel near ticks, xtick={2,3,4,5,6,7}, xticklabels={2,3,4,5,6,7}, ymax=35, every axis plot/.append style={black,thick},cycle list name=myCycleList1] % ybar=0.8pt,bar width=1.5pt, 
\addplot%+[xshift=-2mm] % error bars/.cd,y dir=plus, y explicit, 
coordinates{(2.000,25.926)+-(2.616,2.616) (3.000,29.839)+-(4.038,4.038) (4.000,25.596)+-(5.987,5.987) (5.000,21.503)+-(6.396,6.396) (6.000,19.046)+-(7.053,7.053) (7.000,15.021)+-(6.997,6.997) };
\label{coord_perf_float_symmetric_size_mode:1}
\addplot%+[xshift=-1mm] % error bars/.cd, y dir=plus, y explicit, 
coordinates{(2.000,26.341)+-(2.777,2.777) (3.000,30.4534)+-(1.701,1.701) (4.000,29.360)+-(1.608,1.608) (5.000,27.682)+-(1.896,1.896) (6.000,22.986)+-(4.963,4.963) (7.000,19.536)+-(6.900,6.900) };%
\label{coord_perf_float_symmetric_size_mode:2}
\addplot % error bars/.cd, y dir=plus, y explicit
coordinates{(2.000,4.828)+-(7.495,7.495) (3.000,8.373)+-(6.246,6.246) (4.000,12.796)+-(5.823,5.823) (5.000,14.954)+-(6.246,6.246) (6.000,14.149)+-(5.540,5.540) (7.000,11.905)+-(4.465,4.465) };%
\label{coord_perf_float_symmetric_size_mode:3}
\addplot%+[xshift=1mm] % , error bars/.cd, y dir=plus, y explicit, 
coordinates{(2.000,6.154)+-(0.649,0.649) (3.000,4.639)+-(1.013,1.013) (4.000,5.036)+-(1.143,1.143) (5.000,5.451)+-(0.828,0.828) (6.000,5.167)+-(0.4524,0.4524) (7.000,4.409)+-(0.899,0.899) };%
\label{coord_perf_float_symmetric_size_mode:4}
\addplot%+[xshift=2mm] % , error bars/.cd, y dir=plus, y explicit, 
coordinates{(2.000,6.212)+-(2.248,2.248) (3.000,4.222)+-(1.768,1.768) (4.000,2.938)+-(1.478,1.478) (5.000,2.447)+-(1.488,1.488) (6.000,2.641)+-(1.815,1.815) (7.000,1.618)+-(1.302,1.302) };%
\label{coord_perf_float_symmetric_size_mode:5}
\end{axis}
\end{tikzpicture}
%\caption{
%\footnotesize Dargestellt sind über die Tensorgröße und Kontraktionsmodus gemittelten \textbf{Durchsätze in Gflops} der \textbf{Tensor}"=\textbf{Vektor}"=\textbf{Multiplikation}. \textbf{TLib-SB-P3} \ref{coord_perf_float_symmetric_size_mode:1},\textbf{TLib-LB-P3} \ref{coord_perf_float_symmetric_size_mode:2},\textbf{TCL} \ref{coord_perf_float_symmetric_size_mode:3},\textbf{TBLIS} \ref{coord_perf_float_symmetric_size_mode:4},\textbf{EIGEN} \ref{coord_perf_float_symmetric_size_mode:5}. Daten sind in \textbf{Floating-Point<Single>} codiert.
%\label{fig:ttv_plot_perf_float}
%}
\hfill
\begin{tikzpicture}
\begin{semilogyaxis}[height=0.25\textheight,width=0.5\textwidth,style={font=\footnotesize},grid=major,grid style={dotted},align=center,xlabel={Order},ylabel={Speedup}, xlabel near ticks, ylabel near ticks, xtick={2,3,4,5,6,7}, xticklabels={2,3,4,5,6,7}, ytick={1,10,100}, yticklabels={1,10,100}, ymax=200, every axis plot/.append style={black,thick}, cycle list name=myCycleList2] % ybar=0.8pt,bar width=1.5pt, 
\addplot%+[error bars/.cd,y dir=plus, y explicit] % error bars/.cd,y dir=plus, y explicit, 
coordinates{(2.000,1.018)+-(0.065,0.065) (3.000,1.166)+-(1.112,1.112) (4.000,1.307)+-(0.898,0.898) (5.000,1.448)+-(0.581,0.581) (6.000,1.432)+-(0.681,0.681) (7.000,1.556)+-(0.4514,0.4514) };%
\label{coord_ratio_float_symmetric_size_mode:1}
\addplot%+[xshift=-1mm] % error bars/.cd,y dir=plus, y explicit, 
coordinates{(2.000,14.039)+-(8.223,8.223) (3.000,5.595)+-(3.057,3.057) (4.000,3.007)+-(1.860,1.860) (5.000,2.347)+-(1.566,1.566) (6.000,2.072)+-(1.417,1.417) (7.000,2.004)+-(1.236,1.236) };%
\label{coord_ratio_float_symmetric_size_mode:3}
\addplot%+[xshift=1mm] % error bars/.cd,y dir=plus, y explicit, 
coordinates{(2.000,4.436)+-(1.768,1.768) (3.000,6.868)+-(1.156,1.156) (4.000,6.076)+-(1.116,1.116) (5.000,5.159)+-(0.584,0.584) (6.000,4.534)+-(1.127,1.127) (7.000,4.693)+-(2.103,2.103) };%
\label{coord_ratio_float_symmetric_size_mode:4}
\addplot%+[xshift=3mm] % error bars/.cd,y dir=plus, y explicit, 
coordinates{(2.000,8.283)+-(12.755,12.755) (3.000,23.668)+-(39.378,39.378) (4.000,38.552)+-(66.399,66.399) (5.000,53.890)+-(88.690,88.690) (6.000,55.831)+-(87.581,87.581) (7.000,59.016)+-(78.221,78.221) };%
\label{coord_ratio_float_symmetric_size_mode:5}
\end{semilogyaxis}
\end{tikzpicture}
\vspace{5pt}
%\caption{
%\footnotesize Dargestellt sind über die Tensorgröße und Kontraktionsmodus gemittelten \textbf{Laufzeitverhältnisse} der \textbf{Tensor}"=\textbf{Vektor}"=\textbf{Multiplikation}. Verglichen wurde die Variante \textbf{TLib-LB-P3} mit den obigen Varianten. \textbf{TLib-SB-P3} \ref{coord_ratio_float_symmetric_size_mode:1},\textbf{TCL} \ref{coord_ratio_float_symmetric_size_mode:3},\textbf{TBLIS} \ref{coord_ratio_float_symmetric_size_mode:4},\textbf{EIGEN} \ref{coord_ratio_float_symmetric_size_mode:5}. Daten sind in \textbf{Floating-Point<Single>} codiert.
%\label{fig:ttv_plot_ratio_float}


% 2D-Plots mit Durchschnittswerten gemittelt über die Tensorgröße und Tensorstufe
\begin{tikzpicture}
\begin{axis}[height=0.25\textheight,width=0.5\textwidth,style={font=\footnotesize},grid=major,grid style={dotted},align=center,xlabel={Mode},xlabel near ticks, ylabel near ticks, ylabel={Gflops/s}, xtick={1,2,3,4,5,6,7}, xticklabels={1,2,3,4,5,6,7}, ymax=35, every axis plot/.append style={black,thick},cycle list name=myCycleList1] % ybar=0.8pt,bar width=1.5pt, 
\addplot%+[xshift=-2mm] % error bars/.cd,y dir=plus, y explicit,
coordinates{(1.000,26.744)+-(7.300,7.300) (2.000,14.973)+-(9.609,9.609) (3.000,22.148)+-(5.822,5.822) (4.000,22.734)+-(5.644,5.644) (6.000,24.551)+-(5.710,5.710) (7.000,25.780)+-(2.977,2.977) };%
\label{coord_perf_float_symmetric_size_order:1}
\addplot%+[xshift=-1mm] % error bars/.cd,y dir=plus, y explicit,
coordinates{(1.000,26.986)+-(7.333,7.333) (2.000,23.269)+-(8.966,8.966) (3.000,27.445)+-(2.931,2.931) (4.000,26.860)+-(2.806,2.806) (6.000,26.206)+-(2.550,2.550) (7.000,25.872)+-(2.848,2.848) };%
\label{coord_perf_float_symmetric_size_order:2}
\addplot%+[xshift=0mm] % error bars/.cd,y dir=plus, y explicit,
coordinates{(1.000,19.407)+-(6.152,6.152) (2.000,13.793)+-(6.647,6.647) (3.000,9.619)+-(5.650,5.650) (4.000,8.334)+-(5.059,5.059) (6.000,7.844)+-(4.754,4.754) (7.000,8.006)+-(4.975,4.975) };%
\label{coord_perf_float_symmetric_size_order:3}
\addplot%+[xshift=1mm] % error bars/.cd,y dir=plus, y explicit,
coordinates{(1.000,5.719)+-(0.912,0.912) (2.000,5.835)+-(0.696,0.696) (3.000,5.482)+-(0.964,0.964) (4.000,5.007)+-(0.869,0.869) (6.000,4.351)+-(0.985,0.985) (7.000,4.462)+-(0.830,0.830) };%
\label{coord_perf_float_symmetric_size_order:4}
\addplot%+[xshift=2mm] % error bars/.cd,y dir=plus, y explicit,
coordinates{(1.000,0.326)+-(0.442,0.442) (2.000,3.174)+-(2.734,2.734) (3.000,3.657)+-(2.025,2.025) (4.000,4.031)+-(1.725,1.725) (6.000,4.373)+-(1.529,1.529) (7.000,4.516)+-(1.339,1.339) };%
\label{coord_perf_float_symmetric_size_order:5}
\end{axis}
\end{tikzpicture}
\hfill
%\caption{
%\footnotesize Dargestellt sind über die Tensorgröße und Tensorstufe gemittelten \textbf{Durchsätze in Gflops} der \textbf{Tensor}"=\textbf{Vektor}"=\textbf{Multiplikation}. \textbf{TLib-SB-P3} \ref{coord_perf_float_symmetric_size_order:1},\textbf{TLib-LB-P3} \ref{coord_perf_float_symmetric_size_order:2},\textbf{TCL} \ref{coord_perf_float_symmetric_size_order:3},\textbf{TBLIS} %\ref{coord_perf_float_symmetric_size_order:4},\textbf{EIGEN} \ref{coord_perf_float_symmetric_size_order:5}. Daten sind in \textbf{Floating-Point<Single>} codiert.
%\label{fig:ttv_plot_perf_float}
%}
\begin{tikzpicture}
\begin{semilogyaxis}[height=0.25\textheight,width=0.5\textwidth,style={font=\footnotesize},grid=major,grid style={dotted},align=center,xlabel={Mode}, xlabel near ticks, ylabel={Speedup}, ylabel near ticks, xtick={1,2,3,4,5,6,7},xticklabels={1,2,3,4,5,6,7}, ytick={1,10,100},yticklabels={1,10,100}, ymax=200, every axis plot/.append style={black,thick},cycle list name=myCycleList2] % ybar=0.8pt,bar width=1.5pt, 
\addplot%+[xshift=-3mm] % error bars/.cd,y dir=plus, y explicit, 
coordinates{(1.000,1.011)+-(0.062,0.062) (2.000,2.090)+-(1.257,1.257) (3.000,1.411)+-(0.900,0.900) (4.000,1.246)+-(0.301,0.301) (6.000,1.164)+-(0.496,0.496) (7.000,1.005)+-(0.035,0.035) };%
\label{coord_ratio_float_symmetric_size_order:1}
\addplot%+[xshift=-1mm] % error bars/.cd,y dir=plus, y explicit, 
coordinates{(1.000,1.578)+-(0.930,0.930) (2.000,3.974)+-(6.217,6.217) (3.000,5.564)+-(5.951,5.951) (4.000,5.902)+-(5.921,5.921) (6.000,6.033)+-(5.862,5.862) (7.000,6.014)+-(6.057,6.057) };%
\label{coord_ratio_float_symmetric_size_order:3}
\addplot%+[xshift=1mm] % error bars/.cd,y dir=plus, y explicit, 
coordinates{(1.000,4.788)+-(1.919,1.919) (2.000,3.892)+-(1.349,1.349) (3.000,5.205)+-(1.308,1.308) (4.000,5.582)+-(1.389,1.389) (6.000,6.326)+-(1.465,1.465) (7.000,5.973)+-(1.182,1.182) };%
\label{coord_ratio_float_symmetric_size_order:4}
\addplot%+[xshift=3mm] % error bars/.cd,y dir=plus, y explicit, 
coordinates{(1.000,170.161)+-(83.901,83.901) (2.000,36.612)+-(40.688,40.688) (3.000,11.912)+-(10.405,10.405) (4.000,7.757)+-(2.702,2.702) (6.000,6.646)+-(1.956,1.956) (7.000,6.150)+-(1.589,1.589) };%
\label{coord_ratio_float_symmetric_size_order:5}
\end{semilogyaxis}
\end{tikzpicture}
%\caption{
%\footnotesize Dargestellt sind über die Tensorgröße und Tensorstufe gemittelten \textbf{Laufzeitverhältnisse} der \textbf{Tensor}"=\textbf{Vektor}"=\textbf{Multiplikation}. Verglichen wurde die Variante \textbf{TLib-LB-P3} mit den obigen Varianten. \textbf{TLib-SB-P3} \ref{coord_ratio_float_symmetric_size_order:1},\textbf{TCL} \ref{coord_ratio_float_symmetric_size_order:3},\textbf{TBLIS} \ref{coord_ratio_float_symmetric_size_order:4},\textbf{EIGEN} \ref{coord_ratio_float_symmetric_size_order:5}. Daten sind in \textbf{Floating-Point<Single>} codiert.
%\label{fig:ttv_plot_ratio_float}
%}