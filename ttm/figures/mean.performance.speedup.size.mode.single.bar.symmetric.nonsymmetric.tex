% 2D-Plots mit Durchschnittswerten gemittelt über die Tensorgröße und Kontraktionsmodus für nicht-symmetrisch geformte Tensoren
\begin{tikzpicture}
\begin{axis}[height=0.25\textheight,width=0.55\textwidth, style={font=\footnotesize}, grid=major, grid style={dotted}, align=center, xlabel={Order}, title={Performance [Gflops/s]}, xlabel near ticks, ylabel near ticks, xtick={2,3,4,5,6,7,8,9,10},xticklabels={2,3,4,5,6,7,8,9,10}, ymax=35, every axis plot/.append style={black,thick},cycle list name=myCycleList1] % ybar=0.8pt,bar width=1.5pt, 
%\begin{tikzpicture}
%\begin{axis}[height=0.7\textheight,width=0.7\textwidth,style={font=\footnotesize},grid=major,grid style={dotted},align=center,xlabel={Tensorstufe},ylabel={Durchsatz [Gflops/s]},title={},xtick={2,3,4,5,6,7,8,9,10},xticklabels={2,3,4,5,6,7,8,9,10},ybar,bar width=4pt,every axis plot/.append style={fill},cycle list name=Dark2]
\addplot
coordinates{(2.000,18.412)+-(4.240,4.240) (3.000,19.812)+-(5.033,5.033) (4.000,21.126)+-(4.280,4.280) (5.000,25.513)+-(1.997,1.997) (6.000,30.213)+-(2.522,2.522) (7.000,31.046)+-(1.943,1.943) (8.000,31.361)+-(1.343,1.343) (9.000,32.599)+-(1.739,1.739) (10.000,33.298)+-(1.563,1.563) };%
\label{coord_perf_float_asymmetric_size_mode:1}
\addplot
coordinates{(2.000,17.243)+-(4.825,4.825) (3.000,18.590)+-(5.016,5.016) (4.000,19.893)+-(5.231,5.231) (5.000,22.425)+-(5.888,5.888) (6.000,26.295)+-(4.886,4.886) (7.000,26.596)+-(6.288,6.288) (8.000,25.910)+-(7.636,7.636) (9.000,25.155)+-(7.647,7.647) (10.000,25.332)+-(7.698,7.698) };%
\label{coord_perf_float_asymmetric_size_mode:2}
\addplot
coordinates{(2.000,4.202)+-(6.677,6.677) (3.000,4.027)+-(6.493,6.493) (4.000,1.361)+-(2.078,2.078) (5.000,1.442)+-(2.044,2.044) (6.000,1.536)+-(2.069,2.069) (7.000,1.720)+-(2.076,2.076) (8.000,1.942)+-(2.004,2.004) (9.000,2.243)+-(2.021,2.021) (10.000,2.702)+-(1.975,1.975) };%
\label{coord_perf_float_asymmetric_size_mode:3}
\addplot
coordinates{(2.000,29.063)+-(5.533,5.533) (3.000,27.623)+-(6.985,6.985) (4.000,26.283)+-(7.508,7.508) (5.000,26.040)+-(7.216,7.216) (6.000,28.863)+-(5.083,5.083) (7.000,28.950)+-(6.827,6.827) (8.000,28.544)+-(8.076,8.076) (9.000,28.286)+-(8.124,8.124) (10.000,25.181)+-(10.100,10.100) };%
\label{coord_perf_float_asymmetric_size_mode:4}
\addplot
coordinates{(2.000,6.459)+-(2.069,2.069) (3.000,5.519)+-(1.903,1.903) (4.000,4.886)+-(1.746,1.746) (5.000,3.386)+-(1.200,1.200) (6.000,2.806)+-(1.021,1.021) (7.000,2.435)+-(0.927,0.927) (8.000,2.239)+-(0.857,0.857) (9.000,2.087)+-(0.794,0.794) (10.000,1.972)+-(0.725,0.725) };%
\label{coord_perf_float_asymmetric_size_mode:5}
\end{axis}
\end{tikzpicture}
%\caption{
%\footnotesize Dargestellt sind über die Tensorgröße gemittelten \textbf{Durchsätze in Gflops} der \textbf{Tensor}"=\textbf{Vektor}"=\textbf{Multiplikation}. %\textbf{TLib-SB-P3} \ref{coord_perf_float_asymmetric_size_mode:1},\textbf{TLib-LB-P3} \ref{coord_perf_float_asymmetric_size_mode:2},\textbf{TCL} \ref{coord_perf_float_asymmetric_size_mode:3},\textbf{TBLIS} %\ref{coord_perf_float_asymmetric_size_mode:4},\textbf{EIGEN} \ref{coord_perf_float_asymmetric_size_mode:5}. Daten sind in \textbf{Floating-Point<Single>} codiert.
%\label{fig:ttv_plot_perf_float}
%}
\hfill
\begin{tikzpicture}
\begin{semilogyaxis}[height=0.25\textheight,width=0.55\textwidth,style={font=\footnotesize},grid=major,grid style={dotted},align=center,xlabel={Order},title={Speedup of \ttt{TLib-SB-P3}}, xlabel near ticks, ylabel near ticks, xtick={2,3,4,5,6,7,8,9,10}, xticklabels={2,3,4,5,6,7,8,9,10}, ytick={1,10,100}, yticklabels={1,10,100}, ymax=200, every axis plot/.append style={black,thick}, cycle list name=myCycleList2] % ybar=0.8pt,bar width=1.5pt, 
\addplot
coordinates{(2.000,1.078)+-(0.062,0.062) (3.000,1.067)+-(0.082,0.082) (4.000,1.090)+-(0.151,0.151) (5.000,1.249)+-(0.457,0.457) (6.000,1.221)+-(0.410,0.410) (7.000,1.323)+-(0.677,0.677) (8.000,1.421)+-(0.732,0.732) (9.000,1.526)+-(0.787,0.787) (10.000,1.544)+-(0.787,0.787) };
\label{coord_ratio_float_asymmetric_size_mode:2}
\addplot
coordinates{(2.000,2.048)+-(0.730,0.730) (3.000,2.397)+-(0.897,0.897) (4.000,7.427)+-(2.795,2.795) (5.000,8.163)+-(2.886,2.886) (6.000,9.099)+-(3.460,3.460) (7.000,8.165)+-(3.542,3.542) (8.000,7.119)+-(3.610,3.610) (9.000,6.212)+-(3.539,3.539) (10.000,4.797)+-(2.941,2.941) };
\label{coord_ratio_float_asymmetric_size_mode:3}
\addplot
coordinates{(2.000,0.751)+-(0.718,0.718) (3.000,0.768)+-(0.332,0.332) (4.000,0.877)+-(0.314,0.314) (5.000,1.099)+-(0.455,0.455) (6.000,1.111)+-(0.379,0.379) (7.000,1.215)+-(0.642,0.642) (8.000,1.273)+-(0.671,0.671) (9.000,1.342)+-(0.712,0.712) (10.000,1.777)+-(1.223,1.223) };
\label{coord_ratio_float_asymmetric_size_mode:4}
\addplot
coordinates{(2.000,7.752)+-(15.487,15.487) (3.000,19.915)+-(48.065,48.065) (4.000,34.614)+-(87.775,87.775) (5.000,45.468)+-(110.213,110.213) (6.000,56.685)+-(134.235,134.235) (7.000,65.234)+-(153.266,153.266) (8.000,74.537)+-(175.931,175.931) (9.000,86.830)+-(206.215,206.215) (10.000,96.988)+-(231.323,231.323) };
\label{coord_ratio_float_asymmetric_size_mode:5}
\end{semilogyaxis}
\end{tikzpicture}
%\caption{
%\footnotesize Dargestellt sind über die Tensorgröße gemittelten \textbf{Laufzeitverhältnisse} der \textbf{Tensor}"=\textbf{Vektor}"=\textbf{Multiplikation}. Verglichen wurde die Variante \textbf{TLib-SB-P3} mit den obigen Varianten. \textbf{TLib-LB-P3} \ref{coord_ratio_float_asymmetric_size_mode:2}, \textbf{TCL} \ref{coord_ratio_float_asymmetric_size_mode:3}, \textbf{TBLIS} \ref{coord_ratio_float_asymmetric_size_mode:4}, \textbf{EIGEN} \ref{coord_ratio_float_asymmetric_size_mode:5}. Daten sind in \textbf{Floating-Point<Single>} codiert.
%\label{fig:ttv_plot_ratio_float}
%}
\vspace{5pt}


% 2D-Plots mit Durchschnittswerten gemittelt über die Tensorgröße und Kontraktionsmodus für symmetrisch-geformte Tensoren
\begin{tikzpicture}
\begin{axis}[height=0.25\textheight,width=0.55\textwidth, style={font=\footnotesize}, grid=major, grid style={dotted}, align=center, xlabel={Order}, title={Performance [Gflops/s]}, xlabel near ticks, ylabel near ticks, xtick={2,3,4,5,6,7}, xticklabels={2,3,4,5,6,7}, ymax=35, every axis plot/.append style={black,thick},cycle list name=myCycleList1] % ybar=0.8pt,bar width=1.5pt, 
\addplot%+[xshift=-2mm] % error bars/.cd,y dir=plus, y explicit, 
coordinates{(2.000,25.926)+-(2.616,2.616) (3.000,29.839)+-(4.038,4.038) (4.000,25.596)+-(5.987,5.987) (5.000,21.503)+-(6.396,6.396) (6.000,19.046)+-(7.053,7.053) (7.000,15.021)+-(6.997,6.997) };
\label{coord_perf_float_symmetric_size_mode:1}
\addplot%+[xshift=-1mm] % error bars/.cd, y dir=plus, y explicit, 
coordinates{(2.000,26.341)+-(2.777,2.777) (3.000,30.4534)+-(1.701,1.701) (4.000,29.360)+-(1.608,1.608) (5.000,27.682)+-(1.896,1.896) (6.000,22.986)+-(4.963,4.963) (7.000,19.536)+-(6.900,6.900) };%
\label{coord_perf_float_symmetric_size_mode:2}
\addplot % error bars/.cd, y dir=plus, y explicit
coordinates{(2.000,4.828)+-(7.495,7.495) (3.000,8.373)+-(6.246,6.246) (4.000,12.796)+-(5.823,5.823) (5.000,14.954)+-(6.246,6.246) (6.000,14.149)+-(5.540,5.540) (7.000,11.905)+-(4.465,4.465) };%
\label{coord_perf_float_symmetric_size_mode:3}
\addplot%+[xshift=1mm] % , error bars/.cd, y dir=plus, y explicit, 
coordinates{(2.000,6.154)+-(0.649,0.649) (3.000,4.639)+-(1.013,1.013) (4.000,5.036)+-(1.143,1.143) (5.000,5.451)+-(0.828,0.828) (6.000,5.167)+-(0.4524,0.4524) (7.000,4.409)+-(0.899,0.899) };%
\label{coord_perf_float_symmetric_size_mode:4}
\addplot%+[xshift=2mm] % , error bars/.cd, y dir=plus, y explicit, 
coordinates{(2.000,6.212)+-(2.248,2.248) (3.000,4.222)+-(1.768,1.768) (4.000,2.938)+-(1.478,1.478) (5.000,2.447)+-(1.488,1.488) (6.000,2.641)+-(1.815,1.815) (7.000,1.618)+-(1.302,1.302) };%
\label{coord_perf_float_symmetric_size_mode:5}
\end{axis}
\end{tikzpicture}
%\caption{
%\footnotesize Dargestellt sind über die Tensorgröße und Kontraktionsmodus gemittelten \textbf{Durchsätze in Gflops} der \textbf{Tensor}"=\textbf{Vektor}"=\textbf{Multiplikation}. \textbf{TLib-SB-P3} \ref{coord_perf_float_symmetric_size_mode:1},\textbf{TLib-LB-P3} \ref{coord_perf_float_symmetric_size_mode:2},\textbf{TCL} \ref{coord_perf_float_symmetric_size_mode:3},\textbf{TBLIS} \ref{coord_perf_float_symmetric_size_mode:4},\textbf{EIGEN} \ref{coord_perf_float_symmetric_size_mode:5}. Daten sind in \textbf{Floating-Point<Single>} codiert.
%\label{fig:ttv_plot_perf_float}
%}
\hfill
\begin{tikzpicture}
\begin{semilogyaxis}[height=0.25\textheight,width=0.55\textwidth,style={font=\footnotesize},grid=major,grid style={dotted},align=center,xlabel={Order},title={Speedup of \ttt{TLib-LB-P3}}, xlabel near ticks, ylabel near ticks, xtick={2,3,4,5,6,7}, xticklabels={2,3,4,5,6,7}, ytick={1,10,100}, yticklabels={1,10,100}, ymax=200, every axis plot/.append style={black,thick}, cycle list name=myCycleList2] % ybar=0.8pt,bar width=1.5pt, 
\addplot%+[error bars/.cd,y dir=plus, y explicit] % error bars/.cd,y dir=plus, y explicit, 
coordinates{(2.000,1.018)+-(0.065,0.065) (3.000,1.166)+-(1.112,1.112) (4.000,1.307)+-(0.898,0.898) (5.000,1.448)+-(0.581,0.581) (6.000,1.432)+-(0.681,0.681) (7.000,1.556)+-(0.4514,0.4514) };%
\label{coord_ratio_float_symmetric_size_mode:1}
\addplot%+[xshift=-1mm] % error bars/.cd,y dir=plus, y explicit, 
coordinates{(2.000,14.039)+-(8.223,8.223) (3.000,5.595)+-(3.057,3.057) (4.000,3.007)+-(1.860,1.860) (5.000,2.347)+-(1.566,1.566) (6.000,2.072)+-(1.417,1.417) (7.000,2.004)+-(1.236,1.236) };%
\label{coord_ratio_float_symmetric_size_mode:3}
\addplot%+[xshift=1mm] % error bars/.cd,y dir=plus, y explicit, 
coordinates{(2.000,4.436)+-(1.768,1.768) (3.000,6.868)+-(1.156,1.156) (4.000,6.076)+-(1.116,1.116) (5.000,5.159)+-(0.584,0.584) (6.000,4.534)+-(1.127,1.127) (7.000,4.693)+-(2.103,2.103) };%
\label{coord_ratio_float_symmetric_size_mode:4}
\addplot%+[xshift=3mm] % error bars/.cd,y dir=plus, y explicit, 
coordinates{(2.000,8.283)+-(12.755,12.755) (3.000,23.668)+-(39.378,39.378) (4.000,38.552)+-(66.399,66.399) (5.000,53.890)+-(88.690,88.690) (6.000,55.831)+-(87.581,87.581) (7.000,59.016)+-(78.221,78.221) };%
\label{coord_ratio_float_symmetric_size_mode:5}
\end{semilogyaxis}
\end{tikzpicture}
\vspace{10pt}
%\caption{
%\footnotesize Dargestellt sind über die Tensorgröße und Kontraktionsmodus gemittelten \textbf{Laufzeitverhältnisse} der \textbf{Tensor}"=\textbf{Vektor}"=\textbf{Multiplikation}. Verglichen wurde die Variante \textbf{TLib-LB-P3} mit den obigen Varianten. \textbf{TLib-SB-P3} \ref{coord_ratio_float_symmetric_size_mode:1},\textbf{TCL} \ref{coord_ratio_float_symmetric_size_mode:3},\textbf{TBLIS} \ref{coord_ratio_float_symmetric_size_mode:4},\textbf{EIGEN} \ref{coord_ratio_float_symmetric_size_mode:5}. Daten sind in \textbf{Floating-Point<Single>} codiert.
%\label{fig:ttv_plot_ratio_float}