\section{Summary}
\label{sec:summary}
We have presented efficient layout-oblivious algorithms for the compute-bound tensor-matrix multiplication that is essential for many tensor methods.
Our approach makes use of the LOG method and computes the tensor-matrix product in-place without transposing tensors.
It applies the flexible approach described in \cite{bassoy:2019:ttv} and generalizes the findings on tensor slicing in \cite{li:2015:input} for linear tensor layouts.
The resulting algorithms are capable of processing dense tensors with arbitrary tensor order, dimensions and with any linear tensor layout all of which can be runtime variable.
This degree of flexibility simplifies the integration and application of our library in existing frameworks with different requirements and tensor layouts.

We have presented multiple algorithm variations of the eighth TTM case which either calls a single- or multi-threaded \ttt{cblas\_gemm} with small or large tensor slices in parallel or sequentially.
Additionally, a simple heuristic has been proposed, selecting one of the variants based on the performance evaluation in the original work \cite{bassoy:2024:ttm}. 
We have evaluated all algorithms using a large set of tensor instances with artificial and real-world tensor shapes on an Intel Xeon Gold 5318Y and an AMD EPYC 9354 CPUs.
More precisely, we analyzed the impact of performing the \ttt{gemm} function with subtensors and tensor slices.
Our findings indicate that, subtensors are most effective with symmetrically shaped tensors, irrespective of the parallelization method. 
Conversely, tensor slices are more advantageous when dealing with asymmetrically shaped tensors, particularly when both the contraction mode and leading dimension are large.
Our runtime results demonstrate that parallel executed single-threaded \ttt{gemm} performs best with symmetrically shaped tensors.
If the leading and contraction dimensions are large, functions with a multi-threaded \ttt{gemm} outperforms those with a single-threaded \ttt{gemm} for any type of slicing.
It can also be observed that function \ttt{<combined>} with a simple heuristic performs in most cases as fast as \ttt{<par-gemm,subtensor>} and \ttt{<par-loop,slice>}, depending on the tensor shape.
Function \ttt{<batched-gemm>} is less efficient in case of asymmetrically shaped tensors with large contraction and leading dimensions.
While matrix storage formats have only a minor impact on TTM performance, runtime measurements show that a TTM using MKL on the Intel Xeon Gold 5318Y CPU achieves higher per-core performance than a TTM with AOCL on the AMD EPYC 9354 processor.
We have also demonstrated that our algorithms perform consistently well for $k$-order tensor layouts, indicating that they are layout-oblivious and do not depend on a specific tensor format.

Our runtime tests with other libraries show that TLIB's \ttt{<combined>} version of TTM is either performs equally well or faster than other libraries for the majority of tensor instances.
In case of tensors with artificial tensor shapes, TLIB computes the tensor product at least $12.37$\% faster than LibTorch and Eigen, independent of the processor.
TBLIS and TCL achieve a median throughput that is comparable with TLIB when run on the AMD CPU.
We observed that most libraries are slower than TLIB for the eighth TTM case across the majority of tensor instances, indicating that our proposed heuristic is efficient.
In case of tensors with real-world tensor shapes, TLIB performs better than all libraries for the majority of tensor shapes, reaching a maximum speedup of at least $100.80$\% in some tensor instances.
Exceptions are the CESM-ATM and Miranda data sets where TuckerMPI is $46.8$\% and $13.7$\% faster than TLIB on the Intel CPU.
Also TCL is $16.22$\% and $71.65$\% faster than TLIB when using the NYX and Miranda data sets on the AMD CPU, respectively.


\section{Conclusion and Future Work}
\label{sec:conclusion}
The performance tests show that our algorithms are layout-oblivious and do not need layout-specific optimizations, even for different storage ordering of the input matrix.
Despite the flexible design, our best-performing algorithm is able to outperform Intel's BLAS-like extension function \ttt{cblas\_gemm\_batch} by a factor of up to $2.57$ in case of asymmetrically shaped tensors.
Furthermore, the performance results demonstrate that TLIB computes the tensor-matrix product with asymmetrically shaped tensors on average at least $6.21$\% and up to $334.31$\% faster than TuckerMPI, LibTorch and Eigen.
% Icelake (nonsymm): TLIB is faster for at least 67\% and up to 100.00\% test cases  has a median speedup of at least 6.21\% and up to 334.31\% 
% Icelake (symm): Except for TBLIS, TLIB is faster for at least 57\% and up to 99.07\% test cases and has a median speedup of at least 3.05\% and up to 262.28\%.
% Genoa (nonsymm): TLIB is faster for 74.54\% and 94.75\% the test cases and has a median speedup of at 29.69\% and 117.50\% for LibTorch and Eigen. %TLIB is on average as fast as TBLIS and TCL.
% Genoa (symm): Except for TBLIS, TLIB is faster for at least 57\% and up to 99.07\% test cases and has a median speedup of at least 3.05\% and up to 262.28\%.
Our findings leads us to the conclusion that the LOG-based approach is a viable solution for the general tensor-matrix multiplication, capable of matching efficient GETT-based and TGGT-based implementations.
Hence, other actively developed libraries such as LibTorch and Eigen might benefit from our algorithm design.
Our header-only library provides C++ interfaces and a python module which allows frameworks to easily integrate our library.

In the near future, we intend to incorporate our implementations in TensorLy, a widely-used framework for tensor computations \cite{cohen:2022:tensor.computations, kossaifi:2019:tensorly}.
We also would like to integrate our solution to the TuckerMPI library \cite{ballard:2020:tuckermpi} and investigate the performance of HOSVD algorithms using our approach.
Insights provided in \cite{li:2015:input} could help to further increase the performance.
Additionally, we want to explore to what extend our approach can be applied for the general tensor contractions.

\paragraph{Source Code Availability}
Project description and source code can be found at \tf{\url{https://github.com/bassoy/ttm}}.
The sequential tensor-matrix multiplication of TLIB is part of Boost's uBLAS library.