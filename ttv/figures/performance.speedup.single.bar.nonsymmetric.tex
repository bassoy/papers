% 2D-Plots mit Durchschnittswerten gemittelt über die Tensorgröße und Kontraktionsmodus
\begin{tikzpicture}
\begin{axis}[height=0.25\textheight,width=0.5\textwidth, style={font=\footnotesize}, grid=major, grid style={dotted}, align=center, xlabel={Order}, ylabel={Gflops/s}, xlabel near ticks, ylabel near ticks, xtick={2,3,4,5,6,7,8,9,10},xticklabels={2,3,4,5,6,7,8,9,10}, ymax=35, every axis plot/.append style={black,thick},cycle list name=myCycleList1] % ybar=0.8pt,bar width=1.5pt, 
%\begin{tikzpicture}
%\begin{axis}[height=0.7\textheight,width=0.7\textwidth,style={font=\footnotesize},grid=major,grid style={dotted},align=center,xlabel={Tensorstufe},ylabel={Durchsatz [Gflops/s]},title={},xtick={2,3,4,5,6,7,8,9,10},xticklabels={2,3,4,5,6,7,8,9,10},ybar,bar width=4pt,every axis plot/.append style={fill},cycle list name=Dark2]
\addplot
coordinates{(2.000,18.412)+-(4.240,4.240) (3.000,19.812)+-(5.033,5.033) (4.000,21.126)+-(4.280,4.280) (5.000,25.513)+-(1.997,1.997) (6.000,30.213)+-(2.522,2.522) (7.000,31.046)+-(1.943,1.943) (8.000,31.361)+-(1.343,1.343) (9.000,32.599)+-(1.739,1.739) (10.000,33.298)+-(1.563,1.563) };%
\label{coord_perf_float_asymmetric_size_mode:1}
\addplot
coordinates{(2.000,17.243)+-(4.825,4.825) (3.000,18.590)+-(5.016,5.016) (4.000,19.893)+-(5.231,5.231) (5.000,22.425)+-(5.888,5.888) (6.000,26.295)+-(4.886,4.886) (7.000,26.596)+-(6.288,6.288) (8.000,25.910)+-(7.636,7.636) (9.000,25.155)+-(7.647,7.647) (10.000,25.332)+-(7.698,7.698) };%
\label{coord_perf_float_asymmetric_size_mode:2}
\addplot
coordinates{(2.000,4.202)+-(6.677,6.677) (3.000,4.027)+-(6.493,6.493) (4.000,1.361)+-(2.078,2.078) (5.000,1.442)+-(2.044,2.044) (6.000,1.536)+-(2.069,2.069) (7.000,1.720)+-(2.076,2.076) (8.000,1.942)+-(2.004,2.004) (9.000,2.243)+-(2.021,2.021) (10.000,2.702)+-(1.975,1.975) };%
\label{coord_perf_float_asymmetric_size_mode:3}
\addplot
coordinates{(2.000,29.063)+-(5.533,5.533) (3.000,27.623)+-(6.985,6.985) (4.000,26.283)+-(7.508,7.508) (5.000,26.040)+-(7.216,7.216) (6.000,28.863)+-(5.083,5.083) (7.000,28.950)+-(6.827,6.827) (8.000,28.544)+-(8.076,8.076) (9.000,28.286)+-(8.124,8.124) (10.000,25.181)+-(10.100,10.100) };%
\label{coord_perf_float_asymmetric_size_mode:4}
\addplot
coordinates{(2.000,6.459)+-(2.069,2.069) (3.000,5.519)+-(1.903,1.903) (4.000,4.886)+-(1.746,1.746) (5.000,3.386)+-(1.200,1.200) (6.000,2.806)+-(1.021,1.021) (7.000,2.435)+-(0.927,0.927) (8.000,2.239)+-(0.857,0.857) (9.000,2.087)+-(0.794,0.794) (10.000,1.972)+-(0.725,0.725) };%
\label{coord_perf_float_asymmetric_size_mode:5}
\end{axis}
\end{tikzpicture}
%\caption{
%\footnotesize Dargestellt sind über die Tensorgröße gemittelten \textbf{Durchsätze in Gflops} der \textbf{Tensor}"=\textbf{Vektor}"=\textbf{Multiplikation}. %\textbf{TLib-SB-P3} \ref{coord_perf_float_asymmetric_size_mode:1},\textbf{TLib-LB-P3} \ref{coord_perf_float_asymmetric_size_mode:2},\textbf{TCL} \ref{coord_perf_float_asymmetric_size_mode:3},\textbf{TBLIS} %\ref{coord_perf_float_asymmetric_size_mode:4},\textbf{EIGEN} \ref{coord_perf_float_asymmetric_size_mode:5}. Daten sind in \textbf{Floating-Point<Single>} codiert.
%\label{fig:ttv_plot_perf_float}
%}
\hfill
\begin{tikzpicture}
\begin{semilogyaxis}[height=0.25\textheight,width=0.5\textwidth,style={font=\footnotesize},grid=major,grid style={dotted},align=center,xlabel={Order},ylabel={Speedup}, xlabel near ticks, ylabel near ticks, xtick={2,3,4,5,6,7,8,9,10}, xticklabels={2,3,4,5,6,7,8,9,10}, ytick={1,10,100}, yticklabels={1,10,100}, ymax=200, every axis plot/.append style={black,thick}, cycle list name=myCycleList2] % ybar=0.8pt,bar width=1.5pt, 
\addplot
coordinates{(2.000,1.078)+-(0.062,0.062) (3.000,1.067)+-(0.082,0.082) (4.000,1.090)+-(0.151,0.151) (5.000,1.249)+-(0.457,0.457) (6.000,1.221)+-(0.410,0.410) (7.000,1.323)+-(0.677,0.677) (8.000,1.421)+-(0.732,0.732) (9.000,1.526)+-(0.787,0.787) (10.000,1.544)+-(0.787,0.787) };
\label{coord_ratio_float_asymmetric_size_mode:2}
\addplot
coordinates{(2.000,2.048)+-(0.730,0.730) (3.000,2.397)+-(0.897,0.897) (4.000,7.427)+-(2.795,2.795) (5.000,8.163)+-(2.886,2.886) (6.000,9.099)+-(3.460,3.460) (7.000,8.165)+-(3.542,3.542) (8.000,7.119)+-(3.610,3.610) (9.000,6.212)+-(3.539,3.539) (10.000,4.797)+-(2.941,2.941) };
\label{coord_ratio_float_asymmetric_size_mode:3}
\addplot
coordinates{(2.000,0.751)+-(0.718,0.718) (3.000,0.768)+-(0.332,0.332) (4.000,0.877)+-(0.314,0.314) (5.000,1.099)+-(0.455,0.455) (6.000,1.111)+-(0.379,0.379) (7.000,1.215)+-(0.642,0.642) (8.000,1.273)+-(0.671,0.671) (9.000,1.342)+-(0.712,0.712) (10.000,1.777)+-(1.223,1.223) };
\label{coord_ratio_float_asymmetric_size_mode:4}
\addplot
coordinates{(2.000,7.752)+-(15.487,15.487) (3.000,19.915)+-(48.065,48.065) (4.000,34.614)+-(87.775,87.775) (5.000,45.468)+-(110.213,110.213) (6.000,56.685)+-(134.235,134.235) (7.000,65.234)+-(153.266,153.266) (8.000,74.537)+-(175.931,175.931) (9.000,86.830)+-(206.215,206.215) (10.000,96.988)+-(231.323,231.323) };
\label{coord_ratio_float_asymmetric_size_mode:5}
\end{semilogyaxis}
\end{tikzpicture}
%\caption{
%\footnotesize Dargestellt sind über die Tensorgröße gemittelten \textbf{Laufzeitverhältnisse} der \textbf{Tensor}"=\textbf{Vektor}"=\textbf{Multiplikation}. Verglichen wurde die Variante \textbf{TLib-SB-P3} mit den obigen Varianten. \textbf{TLib-LB-P3} \ref{coord_ratio_float_asymmetric_size_mode:2}, \textbf{TCL} \ref{coord_ratio_float_asymmetric_size_mode:3}, \textbf{TBLIS} \ref{coord_ratio_float_asymmetric_size_mode:4}, \textbf{EIGEN} \ref{coord_ratio_float_asymmetric_size_mode:5}. Daten sind in \textbf{Floating-Point<Single>} codiert.
%\label{fig:ttv_plot_ratio_float}
%}
\vspace{5pt}

% 2D-Plots mit Durchschnittswerten gemittelt über die Tensorgröße und Tensorstufe
\begin{tikzpicture}
\begin{axis}[height=0.25\textheight,width=0.5\textwidth,style={font=\footnotesize},grid=major,grid style={dotted},align=center,xlabel={Mode},xlabel near ticks, ylabel near ticks, ylabel={Gflops/s}, xtick={1,2,3,4,6,7,8,9,10}, xticklabels={1,2,3,4,6,7,8,9,10}, ymax=35, every axis plot/.append style={black,thick},cycle list name=myCycleList1] % ybar=0.8pt,bar width=1.5pt, 
%\begin{tikzpicture}
%\begin{axis}[height=0.7\textheight,width=0.7\textwidth,style={font=\footnotesize},grid=major,grid style={dotted},align=center,xlabel={Kontraktionsmodus},ylabel={Durchsatz [Gflops/s]},title={},xtick={1,2,3,4,6,7,8,9,10},xticklabels={1,2,3,4,6,7,8,9,10},ybar,bar width=4pt,every axis plot/.append style={fill},cycle list name=Dark2]
\addplot
coordinates{(1.000,31.236)+-(1.484,1.484) (2.000,26.440)+-(9.317,9.317) (3.000,27.225)+-(7.733,7.733) (4.000,27.600)+-(6.574,6.574) (6.000,26.701)+-(5.874,5.874) (7.000,26.348)+-(5.788,5.788) (8.000,26.400)+-(5.532,5.532) (9.000,25.971)+-(5.140,5.140) (10.000,25.459)+-(4.675,4.675) };%
\label{coord_perf_float_asymmetric_size_order:1}
\addplot
coordinates{(1.000,30.927)+-(1.621,1.621) (2.000,23.683)+-(7.969,7.969) (3.000,22.252)+-(7.556,7.556) (4.000,21.119)+-(7.063,7.063) (6.000,20.418)+-(6.151,6.151) (7.000,20.228)+-(6.084,6.084) (8.000,21.000)+-(6.790,6.790) (9.000,22.752)+-(6.715,6.715) (10.000,25.057)+-(5.062,5.062) };%
\label{coord_perf_float_asymmetric_size_order:2}
\addplot
coordinates{(1.000,10.561)+-(6.606,6.606) (2.000,1.853)+-(1.151,1.151) (3.000,1.476)+-(0.842,0.842) (4.000,1.275)+-(0.749,0.749) (6.000,1.163)+-(0.714,0.714) (7.000,1.175)+-(0.719,0.719) (8.000,1.196)+-(0.761,0.761) (9.000,1.220)+-(0.773,0.773) (10.000,1.257)+-(0.791,0.791) };
\label{coord_perf_float_asymmetric_size_order:3}
\addplot
coordinates{(1.000,25.827)+-(6.399,6.399) (2.000,25.082)+-(10.010,10.010) (3.000,27.772)+-(8.113,8.113) (4.000,27.608)+-(7.997,7.997) (6.000,27.308)+-(7.551,7.551) (7.000,27.251)+-(7.374,7.374) (8.000,27.353)+-(7.571,7.571) (9.000,29.229)+-(6.465,6.465) (10.000,31.404)+-(1.457,1.457) };
\label{coord_perf_float_asymmetric_size_order:4}
\addplot
coordinates{(1.000,0.143)+-(0.178,0.178) (2.000,3.990)+-(1.481,1.481) (3.000,4.100)+-(1.575,1.575) (4.000,4.208)+-(1.684,1.684) (6.000,4.048)+-(1.761,1.761) (7.000,3.953)+-(1.812,1.812) (8.000,3.869)+-(1.879,1.879) (9.000,3.776)+-(1.944,1.944) (10.000,3.702)+-(2.031,2.031) };
\label{coord_perf_float_asymmetric_size_order:5}
\end{axis}
\end{tikzpicture}
%\caption{
%\footnotesize Dargestellt sind über die Tensorgröße und Tensorstufe gemittelten \textbf{Durchsätze in Gflops} der \textbf{Tensor}"=\textbf{Vektor}"=\textbf{Multiplikation}. \textbf{TLib-SB-P3} \ref{coord_perf_float_asymmetric_size_order:1},\textbf{TLib-LB-P3} \ref{coord_perf_float_asymmetric_size_order:2},\textbf{TCL} \ref{coord_perf_float_asymmetric_size_order:3},\textbf{TBLIS} \ref{coord_perf_float_asymmetric_size_order:4},\textbf{EIGEN} \ref{coord_perf_float_asymmetric_size_order:5}. Daten sind in \textbf{Floating-Point<Single>} codiert.
%\label{fig:ttv_plot_perf_float}
%}
\hfill
\begin{tikzpicture}
\begin{semilogyaxis}[height=0.25\textheight,width=0.5\textwidth,style={font=\footnotesize},grid=major,grid style={dotted},align=center,xlabel={Mode}, xlabel near ticks, ylabel={Speedup}, ylabel near ticks, xtick={1,2,3,4,6,7,8,9,10}, xticklabels={1,2,3,4,6,7,8,9,10}, ytick={1,10,100} ,yticklabels={1,10,100}, ymax=200, every axis plot/.append style={black,thick},cycle list name=myCycleList2] % ybar=0.8pt,bar width=1.5pt, 
\addplot
coordinates{(1.000,1.015)+-(0.119,0.119) (2.000,1.106)+-(0.083,0.083) (3.000,1.265)+-(0.220,0.220) (4.000,1.419)+-(0.507,0.507) (6.000,1.467)+-(0.722,0.722) (7.000,1.470)+-(0.741,0.741) (8.000,1.457)+-(0.816,0.816) (9.000,1.299)+-(0.753,0.753) (10.000,1.021)+-(0.041,0.041) };%
\label{coord_ratio_float_asymmetric_size_order:2}
\addplot
coordinates{(1.000,0.961)+-(0.565,0.565) (2.000,4.771)+-(2.921,2.921) (3.000,5.996)+-(3.264,3.264) (4.000,7.158)+-(3.532,3.532) (6.000,7.638)+-(3.556,3.556) (7.000,7.444)+-(3.457,3.457) (8.000,7.392)+-(3.449,3.449) (9.000,7.174)+-(3.399,3.399) (10.000,6.893)+-(3.377,3.377) };%
\label{coord_ratio_float_asymmetric_size_order:3}
\addplot
coordinates{(1.000,1.323)+-(0.613,0.613) (2.000,1.326)+-(1.138,1.138) (3.000,1.045)+-(0.328,0.328) (4.000,1.145)+-(0.573,0.573) (6.000,1.175)+-(0.770,0.770) (7.000,1.156)+-(0.747,0.747) (8.000,1.182)+-(0.831,0.831) (9.000,1.048)+-(0.775,0.775) (10.000,0.813)+-(0.157,0.157) };%
\label{coord_ratio_float_asymmetric_size_order:4}
\addplot
coordinates{(1.000,417.966)+-(221.570,221.570) (2.000,8.202)+-(4.662,4.662) (3.000,8.212)+-(4.451,4.451) (4.000,8.228)+-(4.373,4.373) (6.000,8.356)+-(4.254,4.254) (7.000,8.610)+-(4.449,4.449) (8.000,9.039)+-(4.784,4.784) (9.000,9.476)+-(5.304,5.304) (10.000,9.934)+-(5.925,5.925) };
\label{coord_ratio_float_asymmetric_size_order:5}
\end{semilogyaxis}
\end{tikzpicture}
%\caption{
%\footnotesize Dargestellt sind über die Tensorgröße und Tensorstufe gemittelten \textbf{Laufzeitverhältnisse} der \textbf{Tensor}"=\textbf{Vektor}"=\textbf{Multiplikation}. Verglichen wurde die Variante \textbf{TLib-SB-P3} mit den obigen Varianten. \textbf{TLib-LB-P3} \ref{coord_ratio_float_asymmetric_size_order:2}, \textbf{TCL} \ref{coord_ratio_float_asymmetric_size_order:3},\textbf{TBLIS} \ref{coord_ratio_float_asymmetric_size_order:4}, \textbf{EIGEN} \ref{coord_ratio_float_asymmetric_size_order:5}. Daten sind in \textbf{Floating-Point<Single>} codiert.
%\label{fig:ttv_plot_ratio_float}
%}